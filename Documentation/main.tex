\documentclass[11pt, a4paper, oneside]{book}

% Um sprache umzustellen
% \usepackage[ngerman]{babel}
\usepackage[english]{babel}

% Restliche Settings einfügen
\usepackage{Setup/settings}

% Einstellungen für Metainformation in PDF-Datei
\hypersetup{pdftitle={ESC241 Pion and Muon lifetime},
            pdfauthor={Till Böhringer, Lucien Käser, Marc Urech, Richard Salnikov}}

% Was im Footer stehen soll, ifoot -> links, cfoot -> mitte
\ifoot{Pion and muon lifetimes}
\cfoot{ESC241}

% Allgemeine weite um alle Figures darauf zu beziehen
\newcommand\Plotwidth{0.8}
\newcommand\Bilderwidth{0.8}

% Setup wie siunix zahlen schreibt
\sisetup{group-separator = {'}, group-digits = integer}

% some stuff to make writing this report easiert
\newcommand{\electron}{$e^{-}$}
\newcommand{\pion}{$\pi^{-}$}
\newcommand{\muon}{$\mu^{-}$}

\begin{document}

% Bitte Titlepage noch bearbeiten falls nötig
\newgeometry{bottom=1cm, top=4cm} % die Abstände von oben und unten korrigieren
\begin{titlepage}
    \setlength{\headheight}{0cm}
	%\centering
	\includegraphics[width=0.45\textwidth]{\thelogofilename}\par\vspace{1cm}
    % \includesvg[width=0.45\textwidth]{\thelogofilename}\par\vspace{1cm}
	
	\centering
	
	{\bfseries\LARGE University of Zurich\par}
	\vspace{0.7cm}
	
	{\Huge\bfseries Pion and muon lifetimes\par}
	\vspace{0.7cm}

	{\LARGE Data analysis 2025 \par Group project IV \par }
	\vfill

    {\large Authors:\par\vspace{0.2cm}}
	{\Large\itshape Till Böhringer\\ \href{mailto:tillnils.boehringer@uzh.ch}{tillnils.boehringer@uzh.ch} \par
	\Large\itshape Lucien Käser\\ \href{mailto:luciendarian.kaeser@uzh.ch}{luciendarian.kaeser@uzh.ch} \par
	\Large\itshape Marc Urech\\ \href{mailto:marcandre.urech@uzh.ch}{marcandre.urech@uzh.ch} \par
	\Large\itshape Richard Salnikov\\ \href{mailto:richardivan.salnikov@uzh.ch}{richardivan.salnikov@uzh.ch} \par}
	\vfill

	
	{\large Lecturer:\par\vspace{0.2cm}}
	{\Large Patrick Owen}
	\vfill
	\vfill

% Bottom of the page
	{\large \today\par}
\end{titlepage}
\restoregeometry % das das restliche Dokument wieder die normale Geometrie hat
\frontmatter

\tableofcontents
\mainmatter

\chapter{Abstract}
% Short summary: goal, method, main results, ... 

\chapter{Introduction}
% what we want to do
In this project, a simplified simulation of an experiment for measuring the lifetimes of charges pions (\pion) and muons (\muon) will be performed.

% problem description
\section{Motivation}
Negatively charged pions (\pion) are composite particles, that consist of a down quark and an up antiquark. They are unstable and decay predominantly to a muon (\muon) and a muon-antineutrino ($\bar{v_{\mu}}$). The muon, a heavier partner of the electron, is also unstable and decays into an electron (\electron), a muon-neutrino ($v_{\mu}$) and an electron antineutrino ($\bar{v_{e}}$). Neglecting any experimental effects, the time distribution of the \electron produced in the decay chain is given by

\begin{equation}
    N(t) = \frac{N_0}{\tau_{\mu} - \tau_{\pi}}  \left[ \exp{-\frac{t}{\tau_{\mu}}} - \exp{-\frac{t}{\tau_{\pi}}} \right]
    \label{eq:decay_chain_equation}
\end{equation}

Where $\tau_{\mu}$ and $\tau_{\pi}$ are the mean lifetimes of the \muon and \pion respectively. A measurement of this time distribution allows extracting times for $\tau_{\mu}$ and $\tau_{\pi}$.

\section{Setup}

The basic elements of the corresponding real experiment are shown in Figure \ref{fig:experimental_setup} negatively charged pion (\pion) is stopped in the third scintillator. The emitted electron (\electron) is then detected in the fifth scintillator. The time difference between the moment the \pion is stopped in the third scintillator and the \electron is detected in the fifth scintillator is recorded. A spectrum of time differences for many such events will allow for an estimation of the half life times of the \pion ($\tau_{\pi}$) and the \muon ($\tau_{\mu}$).

\begin{figure}[htb]
\begin{center}
\includegraphics[scale=0.25]{images/experimental_setup.png}
\end{center}
\caption{Simplified sketch of the setup of the experiment (from the project documentation). A beam containing \pion passes through scintillators 1 and 2 and a piece of plastic to slow them down, such that they stop in scintillator 3. Scintillator 4 is a counter to reject events in which the beam particle was not stopped in scintillator 3. Electron \electron created in the decay chain are detected in scintillator 5. The signal from scintillator 3 starts a clock, the signal from scintillator 5 stops it.}
\label{fig:experimental_setup}
\end{figure}

% goal of the simulation
% short theoretical background

\chapter{Methods}
\section{Simple simulation}
% d\tau_{\pi}ription of the simulation model
% assumptions made
% algorithms or numerical methods used
% software / tools used

For the simple simulation we simulated \num{10000} decays according to equation \ref{eq:decay_chain_equation} and known values from the Particle Data Group \cite{ParticleDataGroup:2024cfk}.

\pion Mean lifetime: \qty{2.6033(0.0005)e-8}{\s} \\
\muon Mean lifetime: \qty{2.1969811(0.0000022)e-6}{\s}

\chapter{Results}
% present key results (plots, tables, values)
% explain findings objectively, without interpretation

\chapter{Discussion}
% interpretation of the results
% compare with theory or expectations
% sources of error, limitations by the simulation

\include{Setup/Verzeichnisse_eigen}

\end{document}